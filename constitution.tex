\documentclass[a4paper, 11pt]{amsart}
\usepackage{graphicx}
\usepackage{booktabs}
\usepackage{setspace}
\doublespacing
\usepackage{array}
\usepackage{caption}
\usepackage[bottom]{footmisc}

\begin{document}
\title{GNU/LINUX club Constitution}
\author{Officers of the club}
\maketitle
\clearpage


\noindent\begin{minipage}{\linewidth}
\centering
\captionof{table}{Cover information}\begin{tabular}{|>{\hspace{0pt}}p{0.298\linewidth}|>{\hspace{0pt}}p{0.698\linewidth}|} 
\hline
\textbf{Title}                                  & GNU/LINUX  CLUB  \\ 
\hline
\textbf{Officers}                          &                                          \\ 
\hline
\textbf{Date created}                           & 09/22/2021   \\ 
\hline
\textbf{Target audience}                         & Open to people from all backgrounds / experience levels. Students with and without programming experience, who embrace the qualities of open software and innovation or want to learn more.                                                                                                      \\ 
\hline
\textbf{GitHub repository link}                        & https://github.com/bc-linux-club/constitution                                                                                                         \\ 
\hline

\hline
\end{tabular}
\end{minipage}













\section{Purpose of the club}
\paragraph{The philosophy behind the club creation, is for it to work as a space for GNU/Linux enthusiasts and programmers of all backgrounds for fun and personal projects, as well as to educate the public about technology and software.
}
\section{Activities}
\begin{itemize}
    \item Introduction to software development and open software
  \item Learning workshops on GNU/Linux and programming.
  \item GNU/Linux virtual machines
  \item GNU/Linux programming
  \item General programming coding competitions
  \item GNU/Linux competitions
\end{itemize}




\section{Meetings}
\paragraph{Meetings will happen on Tuesdays at 5:30 PM at the 105, seminar room (ground floor of the MAC building)}




\section{Administration}
\paragraph{Officers and Chief Officer
    Officers will all have the same level of power and responsibility, however they will alternate specific jobs as they see fit. The particular jobs will vary depending on the situation.
    The Chief Officer is a scapegoat that acts as the lead coordinator and a point of contact as the de jure president.
}
\subsection{Officer responsibilities:}
\begin{itemize}
    \item Being a Team Player
    \item Handling software and other assets
    \item Showing up to meetings
    \item Organizing events
    \item Knowing what GNU/Linux is
    \item Being a nice person
    \item Managing finances
    \item Installing systems and safety measures
    \item Organizing events
    \item Respecting the constitution
    \item Planning and executing events
    \item Budgeting and spending money
    \item Sending and receiving emails on behalf of the club
    \item Recruiting members
    \item Publicity
\end{itemize}
\paragraph{Failure of officers to comply and align with their responsibilities, may lead to impeachment and vote out of the officer member, with simple 2/3 majority officer votes}


\section{Election of officers}
\begin{itemize}
    \item Majority vote
    \item 3 officers, 1 chief officer
    \item The chief officer will be the de facto president, and the other officers will arbitrarily be assigned to the de facto roles of vice president, secretary, and treasurer.
    \item Any member can vote
    \item Officers can be voted off by majority vote
\end{itemize}

\section{Platforms}
\begin{itemize}
    \item Github organization
    \item The Arch Linux Wiki
    \item Website (github)
    \item https://tldp.org (The Linux Documentation Project)
    \item Unix shell Bash platform (For shell scripts)
\end{itemize}

\section{[non-discrimination policy]}
\paragraph{“The principle of non-discrimination seeks 'to guarantee that human rights are exercised without discrimination of any kind based on race, colour, sex, language, religion, political or other opinion, national or social origin, property, birth or other status such as disability, age, marital and family status, sexual orientation and gender identity, health status, place of residence, economic and social situation.'"}


\end{document}
