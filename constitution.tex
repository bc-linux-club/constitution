\documentclass[a4paper, 11pt]{amsart}
\usepackage{graphicx}
\usepackage{booktabs}
\usepackage{setspace}
\doublespacing
\usepackage{array}
\usepackage{caption}
\usepackage[bottom]{footmisc}

\begin{document}
\title{GNU/LINUX club Constitution}
\author{Officers of the club}
\maketitle
\clearpage


\noindent\begin{minipage}{\linewidth}
\centering
\captionof{table}{Cover information}\begin{tabular}{|>{\hspace{0pt}}p{0.298\linewidth}|>{\hspace{0pt}}p{0.698\linewidth}|} 
\hline
\textbf{Title}                                  & GNU/LINUX  CLUB  \\ 
\hline
\textbf{Officers}                          &    Orion Musselman, Dimitrios Ntentia, Rodolfo Alvarado                                      \\ 
\hline
\textbf{Date created}                           & 09/22/2021   \\ 
\hline
\textbf{Target audience}                         & Open to people from all backgrounds. Students with and without programming experience, who embrace the qualities of open software and innovation.                                                                                                      \\ 
\hline
\textbf{GitHub repository link}                        &     https://github.com/bc-linux-club                                                                                                    \\ 
\hline

\hline
\end{tabular}
\end{minipage}













\section{Purpose of the club:}
\paragraph{The philosophy behind the club creation, is for it to work as a space for GNU/Linux enthusiasts and programmers for fun and personal projects, as well as to educate the public about technology and software that exists out there.
}
\section{Activities:}
\begin{itemize}
    \item Introduction to software development and open software
  \item Learning workshops on GNU/Linux and programming.
  \item GNU/Linux virtual machines
  \item GNU/Linux programming
  \item General programming coding competitions
  \item GNU/Linux competitions
\end{itemize}




\section{Meetings}
\paragraph{Meetings will happen on Sunday at 3:30 PM at the 105, seminar room in ground floor MAC building)}




\section{Administration:}
\paragraph{Officers and Chief Officer
	Officers will all have the same level of power and responsibility, however they will alternate specific jobs as they see fit.
	The Chief Officer is a scapegoat that acts as a coordinator and a point of contact as the de jure president.
}
\subsection{Officer responsibilities:}
\begin{itemize}
    \item Team player
    \item Handle software and other assets
    \item Show up to meetings
    \item Organize events
    \item Know what GNU/Linux is
    \item Be a nice person
    \item Manage finances
    \item In charge for installations and safety measures
    \item  Organizing events
    \item Respect the constitution
\end{itemize}
\paragraph{Failure of officers to comply and align with their responsibilities, may lead to impeachment and vote out of the officer member, with simple 2/3 majority officer votes}


\section{Election of officers}
\begin{itemize}
    \item Majority vote
    \item 3 officers, 1 chief officer
    \item Any member can vote
    \item Officers can be voted off by majority vote
\end{itemize}

\section{Platforms}
\begin{itemize}
    \item Github organization

    \item Website (github)
    \item https://tldp.org (The linux documentation Project)
    \item Unix shell Bash platform(For shell scripts)
\end{itemize}

\section{[non-discrimination policy]}
\paragraph{“The principle of non-discrimination seeks “to guarantee that human rights are exercised without discrimination of any kind based on race, colour, sex, language, religion, political or other opinion, national or social origin, property, birth or other status such as disability, age, marital and family status, sexual orientation and gender identity, health status, place of residence, economic and social situation”.}


\end{document}
